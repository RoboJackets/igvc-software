\documentclass[letterpaper,12pt,notitlepage]{report}

\usepackage[latin1]{inputenc}
\usepackage{amsfonts}
\usepackage{amsmath}
\usepackage{amssymb}
\usepackage{fontenc}
\usepackage{graphicx}

\begin{document}
\title{OSMC Interface Shield}
\author{}
\date{}
\maketitle

\subsection{TODO}
\begin{enumerate}
 \item finish schematic - check use MISO for SS, check for vcc loop on OSMC connetor
 \item finish parts
 \item add decoupling caps (10u, 100u)
 \item initial check
 \item calculate component values for isolators, filters, and decoupling caps
 \item final check - trace widths, emf compatablity, ground plane penetration, etc
 \item cleanup - silkscreen
 \item finish documentation - add image of schematic and layout - legal check
\end{enumerate}

\subsection{Board Description}

\subsection{Pinout}

inverted signal?

\subsection{Design}

Since the sampling rate for the arduino is ? the cutoff for the anti-aliasing filter was chosen to be ?, which makes the sampling rate equal to 10(?) times the maximum frequency.

Optoisolators were used to seperate the two power sources.  Ground planes where used to reduce noise.

This design assumes that the motors are being driven from the same source.

\textquotedblleft The input lines of the 4081A are 'modified TTL' in that a high level signal is any voltage between 3 and 12V\textquotedblright

No header for a joystick was added because there will always be a computer present.

\subsection{Parts list}
\subsubsection{Optoisolators}
\begin{description}
 \item[MOCD213-M] x 3

2 channel optocoupler with transistor output.  Any kind of optocoupler can be used for the digital signals.  The part I picked was the cheapest I could find at the time on Digikey.

 \item[HCNR200/1] x 1

Analog optocoupler for measuring battery voltages.  It would be nicer if this was in a smaller package.

\end{description}

\subsubsection{Headers}
\begin{description}
 \item[?] x 3

for arduino

 \item[?] x 2

for OSMC

 \item[?] x 2

for encoders

 \item[?] x 2

for current sensors

 \end{description}

\subsection{Future}
\begin{itemize}
 \item state of E-stop
 \item joystick?
 \item DC voltage control: using PWM to control the OSMC swtiching is not as efficient as using direct DC voltages.  This could be done by adding a lowpass filter to the PWM output or by using a DAC.
 \item Temperature sensing
 \item Regenerative Motor Braking (part of the OSMC?)
\end{itemize}

\end{document}
